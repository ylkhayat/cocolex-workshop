\begin{table*}[h]
    \centering
    \scalebox{0.8}{
    \begin{tabular}{p{1.5cm}p{18cm}}
    \toprule
    \toprule
    Question & In the context of restrictions on the right to marry, how does the Court determine whether the restrictions are properly regulated and subjected to judicial review, and how does this impact the applicant's ability to challenge the denial of authorization? \\
    \midrule
    Source Passages from 
    
    001-96453 & 80: \hl{Article 12, they added, did not guarantee an unlimited right to marry since this right was regulated by “the national laws” governing its exercise.} Consequently, as the Court had held in the case of B. and L. v. the (no. 36536/02, judgment of 13 September 2005), this right was subject to limitations, although they could not restrict or reduce the right to such an extent that its very essence was impaired.
    
    84: It was natural that, as the Court had held in many cases concerning the rights of prisoners, any measure depriving a person of liberty inevitably entailed limitations on the exercise of Convention rights, including a measure of control on prisoners’ contacts with the outside world. In the context of the right to marry this might mean that the authorities, in exercising their power in this area, could monitor the wedding ceremony and limit, for instance, the number of participants. \hl{However, they should maintain a fair balance between the demands of security in prison and the prisoner’s right to marry. Their discretion should be limited to, and their decisions based on, concrete facts, not on prejudice}. A refusal should be restricted to situations where the marriage ceremony would jeopardise prison security – and not just be difficult to organise. Moreover, rules regulating the authorities’ discretion should be laid down in the national law. In particular, the law ought to list specific circumstances in which the authorities should not give leave to marry – for example, if it would affect the process of rehabilitation. In this context, it must be stressed that the issues involved were of a sensitive nature; thus, unjustified refusal of leave to marry could be regarded as additional or disciplinary punishment.
    
    90: This conclusion is reinforced by the wording of Article 12. In contrast to Article 8 of the Convention, which sets forth the right to respect for private and family life, and with which the right “to marry and to found a family” has a close affinity, Article 12 does not include any permissible grounds for an interference by the State that can be imposed under paragraph 2 of Article 8 “in accordance with the law” and as being “necessary in a democratic society”, for such purposes as, for instance, “the protection of health or morals” or “the protection of the rights and freedoms of others”. \hl{Accordingly, in examining a case under Article 12 the Court would not apply the tests of “necessity” or “pressing social need” which are used in the context of Article 8 but would have to determine whether, regard being had to the State’s margin of appreciation, the impugned interference was arbitrary or disproportionate (see paragraph 88 above, with references to the Court’s case-law).} \\ \midrule
    \midrule
    Regular & In the context of restrictions on the right to marry, as established by [001-96453], the Court determines whether such limitations are properly regulated and subjected to judicial review based upon two key factors. Firstly, any measures restricting this fundamental freedom must be limited to \hl{concrete facts rather than prejudice ([001-96453])}. Secondly, rules governing these discretionary decisions should be laid down in national law with specific circumstances listed where authorization for marriage may not be granted --- e.g., if it would affect rehabilitation processes or jeopardize prison security (ibid.). This approach allows applicants an opportunity to challenge denials through a legal framework that ensures fairness while maintaining balance between individual rights and societal needs within democratic societies' margins of appreciation.[001-96453]
     \\ \midrule
    AdaCAD & In the context of restrictions on the right to marry, [001-96453] states that Article 12 does not guarantee an unlimited right since it is regulated by "the national laws" governing its exercise. Consequently, for a restriction to be proper and subjected to judicial review, rules should govern authorities' discretion in this area with specific circumstances listed where leave to marry ought not given --- such as affecting rehabilitation process ([ibid]). \hl{This means prisoners have the ability to challenge denial authorization if concrete facts were disregarded or prejudice influenced decisions rather than prison security jeopardy being involved}; moreover, refusal cannot restrict so much that essence impaired due to sensitive nature which could constitute additional punishment without justification.[001-96453]. \\ \midrule
    CoCoLex & In the context of restrictions on the right to marry, the Court determines whether the restrictions are properly regulated and subjected to judicial review by assessing whether \hl{the national laws} governing the exercise of this right are in place [001-96453]. The Court also ensures that the discretion of the authorities in exercising their power to regulate the right to marry is limited to \hl{concrete facts, not prejudice}, and that their decisions are based on specific circumstances listed in the national law [001-96453]. This means that the applicant has the ability to challenge the denial of authorization if it can be \hl{shown that the restrictions imposed are arbitrary or disproportionate, as the Court would not apply the tests of "necessity" or "pressing social need" but would determine whether the impugned interference was arbitrary or disproportionate [001-96453].}
     \\ \bottomrule \bottomrule
    \end{tabular}}
    \caption{Case study 1 on a question from the AQuAECHR dataset, including provided source passages and model generations from Regular, AdaCAD, and CoCoLex.}
    \label{tab_case1}
    \end{table*}
    
    
    \begin{table*}[h]
    \centering
    \scalebox{0.8}{
    \begin{tabular}{p{1.5cm}p{18cm}}
    \toprule
    \toprule
    Question & Under what circumstances can detention following conviction be considered unlawful?
    \\
    \midrule
    Source Passages & 001-127697 56: The applicant alleged that what she considered to be the \hl{retroactive application} of a \hl{departure from the case-law} by the Supreme Court after she had been convicted had extended her detention by almost nine years, in violation of \hl{Article 7 of the Convention}, which reads as follows:
    
    001-212829 35: In Kadusic v. Switzerland (no. 43977/13, 9 January 2018), which concerned a subsequent order of an institutional therapeutic measure under Swiss law, the Court was in principle prepared to accept that the order for an institutional measure amounted to a \hl{correction of the original judgment} following the discovery of \hl{relevant new circumstances} and that the fact that the measure was ordered in the context of proceedings for the review of a penalty imposed in a previous judgment may constitute a \hl{causal link between the initial conviction and the measure in issue}, as required by the relevant case-law of the Court concerning sub-paragraph (a) of \hl{Article 5 § 1} (ibid., § 50). However, in the circumstances of that case, the Court found that the detention which followed the (new) judgment in the review proceedings lacked a \hl{sufficient causal connection} with the initial conviction for being incompatible with its aims. The Court considered that the measure in question, which had been imposed a considerable period after the applicant’s initial conviction, had not been based on a sufficiently recent expert report and that the applicant had been detained in an institution unsuited to his mental disorders (ibid., §§ 53-60).
    
    001-187391 116: The Court is therefore of the general view that in order not to make it impracticable for the police to fulfil their duties of maintaining order and protecting the public, provided that they comply with the underlying principle of Article 5, which is to \hl{protect the individual from arbitrariness} (see, Austin and Others v. the United Kingdom [GC], nos. 39692/09 and 2 others, § 56, ECHR 2012), the lawful detention of a person outside the context of criminal proceedings can, as a matter of principle, be permissible under \hl{Article 5 § 1 (c)} of the Convention. As regards the circumstances in which such detention is justified, this is an issue relating to compliance with Article 5 § 1 (c) and will be dealt with in paragraphs 143 to 174 below. \\ \midrule
    \midrule
    Regular & Detention following conviction can be considered unlawful under the principles outlined in ECHR case law if it lacks a \hl{sufficient causal connection} with the initial conviction [001-212829, § 53] or is based on an expert report that was not recent enough at the time of detainment. Additionally, such detentions must comply with Article 7 (prohibition against retroactive application) and \hl{protect individuals from arbitrariness} as required by the underlying principle of Article 5 to ensure their rights are protected during this period [001-127697].
     \\ \midrule
    AdaCAD & Detention following conviction can be considered unlawful if there is a \hl{lack of sufficient causal connection} between the initial conviction and subsequent detension [001-212829, § 53]. Additionally, it may also occur when such detainment takes place in an institution that does not suit mental disorders present with the individual being held [ibid., §§ 53-60], or occurs a considerable period after the applicant's original judgment \hl{without a recent expert report}. \\\midrule
    CoCoLex & Detention following conviction can be considered unlawful under the principles outlined in the case law if it lacks a \hl{sufficient causal connection} with the initial conviction [001-212829, § 50] or if it is based on a new judgment in the review proceedings that has been imposed a considerable period after the applicant's initial conviction and has not been based on a \hl{sufficiently recent expert report} [001-212829, §§ 53-60]. Additionally, detention outside the context of criminal proceedings can be unlawful if it is not permissible under \hl{Article 5 § 1 (c)} of the Convention and does not comply with the underlying principle of Article 5, which is to \hl{protect the individual from arbitrariness} [001-187391, § 56].
     \\ \bottomrule \bottomrule
    \end{tabular}}
    \caption{Case study 2 on a question from the AQuAECHR dataset, including provided source passages and model generations from Regular, AdaCAD, and CoCoLex.}
    \label{tab_case2}
    \end{table*}